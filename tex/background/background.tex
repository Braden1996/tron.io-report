\documentclass[class=article, crop=false]{standalone}

%\addbibresource{refs.bib}

\begin{document}
	"Your background section should end with a clear statement of the research questions problem your project is trying to answer. These will reflect the aim of your project, but will be different in that they explain the problem you are attempting to solve, e.g.,"

	\section{The Game of Tron}
		The game of Tron first emerged in response to a scene from the 1982 film bearing the same name, Tron \parencite{TronLightCycleBattle}. To help avoid ambiguity we shall first describe, in detail, the rules of the game; as they have been understood for this project.

		Tron is a free-for-all game which can be played by two or more players. Players are spawned as a square, the size of one unit, and are uniformly distributed along the four sides of a square which enclose the game's arena. Initially, each player is \emph{directed} inward of the square.

		All players are perpetually in motion, travelling around the arena at the same constant speed. However, at any time, they're able to direct themselves either 90\si{\degree} left or 90\si{\degree} right.

		As a player travels, they leave a wall which occupies the entire area that they, at any time, had covered. We call this a \emph{trail}, and can be thought of as the travelled path.

		A player is considered \emph{dead} if they collide with: the borders of the arena, another player, or the trail of any player (including themselves).

		The game is considered to be finished when there are less than two players alive. In the case where a single player is left alive, they are considered to be the \emph{winner}. Otherwise, the game is a \emph{draw}.

	\section{Existing Games}
		Talk about some of the existing browser based games. Maybe some brief history and the direction in which the standard is moving. For example, modern web-sites (web-applications) share many of the mechanics of a video-game. (real-time communication, expensive processing, prediction)

	\section{Technologies}
		Discuss and justify all the many technologies which were considered during development. Mention things such as browser support and ease-of-use.

	\section{Game Specifics}
		The techniques which were used to implement the game.

		\subsection{Game}
			Such as spatial data-structures

		\subsection{Networking}
			snapshot-based networking, client prediction, lag compensation

		\subsection{AI}
			alpha-beta minimax, monte-carlo search tree
\end{document}