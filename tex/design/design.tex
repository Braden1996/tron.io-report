\documentclass{standalone}

\begin{document}
	Firstly, throughout development Git has been used for version control; the repository is hosted on GitHub\parencite{TronGitRepo}. A todo-list full of various development tasks exists within the root directory of the repository. This helped to maintain an efficient workflow and to document thoughts for later development sessions. Efforts were also made to document and lint all produced source-code.

	The produced implementation consist of many aspects, most of which can be categorised into one of three groups: game mechanics, network communications and artificial intelligence. Thus, where possible, we shall try to discuss each one of these categories separately. However, before that we shall first provide an overview of the application and describe the relation between some of its core components.

	As the game is to be played in a web-browser, we will be making use of the three fundamental web languages, those being HTML, CSS, and JavaScript. These three languages, and the standards which govern them, allow us to confidently write portable code which will run identically within the majority of web-browsers. Sadly, this isn't always the case as many browsers do not support the most up-to-date standard; assuming it follows it in the first place...

	Now, the majority of our game, for the client-side at least, is programmed in JavaScript; allowing it to run in web-browsers. However, due to the compatibility issues mentioned above, we have wrote our game according to the ECMAScript2015 standard \parencite{ECMAScript2015} and transpile it to a more commonly supported syntax using Babel\parencite{Babel}.

	Transpiling to JavaScript has become somewhat the norm in the realm of modern front-end web-development. In recent years, many new programming languages have in fact emerged under the sole purpose of being transpiled into JavaScript. The reason we chose not to use one of these new languages is because it masks away a lot of the underlying code which the browser is expected to run; creating another layer for things to potentially go wrong.

	As the game was to be built for a web-browser, we already have access to a wide range of graphical components, as per the HTML specification. These are useful for components such as buttons and lists. However, it is quite cumbersome to maintain a proper reflection of our game's state (which resides within the JavaScript environment) onto our DOM. This is quite a frequent issue with web-applications and is the basis for many very popular frameworks. With this in mind, along with desire to learn a new framework and the curiosity to find out as to how helpful it would be with this project; React\parencite{React} and Redux\parencite{Redux} have been used to help handle the view layer.

	Given that we already have to use Javascript for our front-end game code, it makes sense to not bring in another language for our back-end. This has a myriad of benefits including not having to deal with the additional quirks of a separate programming language and eliminating the hassle of maintaining two separate implementations of code that is identical in purpose. Hence for the back-end, on the server, we use Node.js\parencite{NodeJs}.

	Somewhere early on during development, having already settled on the aforementioned technologies, it began more and more difficult to maintain an efficient work-flow whilst developing. That inspired the decision to transition the project to use the React Universally \parencite{ReactUniversally} starter-kit; a boilerplate project with a bare-bones project base structure and various scripts to help automate the tedious build/watch process.

	React Universally also sets up Server-side rendering with React. This is helpful as it enables the server to perform an initial render of our application's components, then serve the result to the client. This makes the web-page appear to load faster, as the client is no longer required to perform this initial render themselves; they just mount the components then rehydrate the state.

	The source code is split into three main directories: \emph{shared}, \emph{client}, and \emph{server}. The \emph{shared} directory contains the bulk of our application, including source which is rendered server-side to be served to a client. Whilst the \emph{client} directory is for browser specific source that will not be used server-side or for the purpose of server-side rendering. Similarly, \emph{server} is for source which is executed on our node server.

	\subsection{Game Mechanics}
		The source-code for Tron's game mechanics exist primarily within the \emph{game} sub-directory inside \emph{shared}. There have been attempts to keep it as decoupled as possible from the various libraries we've brought in. This was done to help reduce the risk of delay caused by a library that isn't explicitly required for the main goal of the application.

		The entire game state is stored within a single mutable JavaScript object. There exist numerous operations which can be applied to the state, each in the form of a function that take the state and additional parameters, if applicable. For the case when mutability is not desired, there exists a \emph{copyState} operation which explicitly copies the entire state and rebuilds the cache.

		Below is a breakdown of the structure for our game state object:
		\begin{description}
        \item[tick]: the number of times this state has ticked; inclusive of the current tick update.
        \item[progress]: the amount of time which has passed since the last tick update.
        \item[started]: a boolean indicating if a round of Tron is in progress.
        \item[finished]: a boolean indicating if the current round has finished. \emph{undefined} if started is \emph{false}.
        \item[arenaSize]: the number of cells within our game arena.
        \item[playerSize]: the number of cells a player occupies within our game arena.
        \item[speed]: the number of cells each player travels over the course of a millisecond.
        \item[players]: an array containing an object for each player part of the current game. The following describes the structure of a single player object:
        \begin{description}
        	\item[id]: a unique string used to identify the player.
        	\item[name]: an arbitrary string for other players to recognise this player.
        	\item[alive]: a boolean indicating if the player is alive, or otherwise dead.
        	\item[direction]: the direction (north, south, east, or west) in which the player is travelling.
        	\item[position]: a \emph{point} (an array, in the form of [\emph{x, y}]), representing the player's current position in the grid.
        	\item[trail]: an array of \emph{points} at which the player has changed direction. Used to construct a path of the area in which the player has travelled. However, the array has two special cases: the first element is the player's spawn point, and the last element is the point for the player's previous position.
      	\end{description}
        \item[cache]: a nested object containing various cache structures required by our state. The following describes the structure of the cache object:
        \begin{description}
        	\item[collisionStruct]: a data-structure in which we can check for player/trail collisions within our arena. See for more detail \fullref{collisionDetection}.
      	\end{description}
    \end{description}

		\subsubsection{Game Loop}
			The game loop is charged with updating our game state and redrawing the graphics upon every tick

		\subsubsection{Collision Detection} \label{collisionDetection}

	\section{Game Specifics}
		There are a near endless range of techniques that can be utilised for game development. Here we shall introduce some of the key techniques

		\subsection{Game}
			Such as spatial data-structures

		\subsection{Networking}
			snapshot-based networking, client prediction, lag compensation

		\subsection{AI}
			alpha-beta minimax, monte-carlo search tree
\end{document}