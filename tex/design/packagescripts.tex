\begin{lstlisting}[language=bash,style=codefigure,
caption={Creates an \texttt{webpack-bundle-analyze} session against the production build of the client bundle.}]
yarn run analyze:client
\end{lstlisting}

\begin{lstlisting}[language=bash,style=codefigure,
caption={Creates an \texttt{webpack-bundle-analyze} session against the production build of the server bundle.}]
yarn run analyze:server
\end{lstlisting}

\begin{lstlisting}[language=bash,style=codefigure,
caption={Builds the client and server bundles, with the output being optimized.}]
yarn run build
\end{lstlisting}

\begin{lstlisting}[language=bash,style=codefigure,
caption={Builds the client and server bundles, with the output including development related code.}]
yarn run build:dev
\end{lstlisting}

\begin{lstlisting}[language=bash,style=codefigure,
caption={Deletes any build output that would have originated from the other commands.}]
yarn run clean
\end{lstlisting}

\begin{lstlisting}[language=bash,style=codefigure,
caption={Deploys your application to \href{https://zeit.co/now}{\texttt{now}}.}]
yarn run deploy
\end{lstlisting}

\begin{lstlisting}[language=bash,style=codefigure,
caption={Starts a development server for both the client and server bundles.}]
yarn run develop
\end{lstlisting}

\begin{lstlisting}[language=bash,style=codefigure,
caption={Executes \texttt{eslint} against the project.}]
yarn run lint
\end{lstlisting}

\begin{lstlisting}[language=bash,style=codefigure,
caption={Executes the server. It expects you to have already built the bundlesusing the \texttt{yarn run build} command.}]
yarn run start
\end{lstlisting}

\begin{lstlisting}[language=bash,style=codefigure,
caption={Runs the \texttt{jest} tests.}]
yarn run test
\end{lstlisting}

\begin{lstlisting}[language=bash,style=codefigure,
caption={Runs the \texttt{jest} tests and generates a coverage report.}]
yarn run test:coverage
\end{lstlisting}