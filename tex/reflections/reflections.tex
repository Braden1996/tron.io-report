\documentclass{standalone}

\begin{document}
	When working on a new project, it is expected there will be times of confusion and challenge. These moments are often demoralizing, and can impede in one's motivation to work. Ironically, they are also the times in which \emph{learning} occurs. For something to be worth doing, it must hold some reward. One of the most valuable rewards, especially to a beginner, is the attainment of new skills. Hence, for a project to be worthwhile, it must bear many challenges. This idea is especially prevalent in software development. Almost every project will expose the developer to new ideas. Often these ideas are expressed through the use of a new library or design pattern. However, the developer will also acquire skills that are inexplicit in nature. These skills are often called \enquote{meta-skills}, referring to the cognitive strategies that an individual applies to the processing of new information. They are often equated to experience, and impact the way we approach future projects. Throughout this chapter, we reflect upon our approach to this project, scrutinizing the assumptions that we made. We do this to help better develop our ability as an individual to learn, and tackle challenges that we may later face.

	The realisation that the project exists purely as a learning exercise did, undeniably, encourage the selection of certain decisions. Although some of these decisions may not have been entirely suitable for a commercially motivated project, they did provide a wealth of new skills. One decision, in particular, was in regards to the somewhat excessive use of third-party libraries and techniques. Many of such were employed with little purpose other-than to understand their promoted ideas and intricacies; that are frequently appreciated for being well-designed. However, the decision did introduce the concept of \enquote{technical debt}. This resulted from the plethora of issues solely related to the arguably unnecessary intricacies we had subjected ourselves to. Unfortunately, this snatched time away from the core aspects of the project. Admitting that this was an entirely bad decision, would not be entirely true.

	Another mistake made was the lazy assumption that techniques utilised by similar projects would transfer well onto our own. Not only did this turn out not to be the case, but resulted in greater amounts of work due to the inevitable transition to a more suitable technique. If more time was spent researching and testing a particular technique, we would have most probably saved time overall.

	Despite their hindrance to the final state of our project, it is not entirely unfortunate that these mistakes occurred. They unlocked the opportunity for us to recognise them, understand them, and to build on them going forward. The experiences they taught are an invaluable asset, transferable to many aspects in life, and will evolve our mindset going forward with future endeavours.
\end{document}